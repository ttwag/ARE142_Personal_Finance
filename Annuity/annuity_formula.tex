\documentclass{article} % Defines the document class, article is commonly used
\usepackage[shortlabels]{enumitem}
\usepackage{amsmath}    % Allows for more advanced math formatting
\usepackage{amssymb}    % Provides additional mathematical symbols
\usepackage{amsthm}     % \qed
\usepackage{graphicx}   % image
\usepackage{float}      % image placement
\usepackage{hyperref}
\hypersetup{
    colorlinks=true,       % false: boxed links; true: colored links
    linkcolor=black,       % color of internal links
}
\usepackage[margin=1.5in]{geometry}
\usepackage{siunitx}

\begin{document}

\title{ARE142: Annuity Formula}
\author{Tao Wang}
\date{\today}

\maketitle

\section{Introduction}
\begin{quote}
    There are three types of personal savings strategies in ARE142, each has its own formula.
    Some of them revolve around the concept of \textit{annuity}, which is a contract that provides a series of regular payments over a period.


    They are:

    \begin{enumerate}
        \item One-Time payment now and compound after n years
        \item Save money every year and compound after n years
        \item Withdraw \$X annually for n years from an initial saving that compound
    \end{enumerate}

    They correspond to three formula (1.2, 1.3, 1.4) to calculate their future and present values:

    \subsection{Definition}
    \begin{align}
         & \text{FV = Future Value}                         \\
         & \text{PV = Present Value}                        \\
         & \text{PMT = Periodic Payment in an Annuity}      \\
         & \text{i = Interest Rate}                         \\
         & FV_{year = 0} = PV                               \\
         & FV_{year = k + 1} = FV_{year = k} \times (1 + i) \\
    \end{align}

    \subsection{Future value of initial savings compounded after n years}
    \subsubsection{Formula}
    \begin{quote}
        \[
            FV_{year = n} = PV (1 + i)^n
        \]
    \end{quote}
    \subsubsection{Proof}
    \begin{quote}
        We will show that this is true through the strong induction.

        At year 0, we have
        \[
            FV_{year = 0} = PV (1 + i)^{0} = PV
        \]
        This is true because future value at year 0 is the present value.

        We can assume that right after year k, we have
        \[
            FV_{year = k} = PV (1 + i)^{k}
        \]

        Right after year k + 1, by we have

        \[
            FV_{year = k + 1} = PV (1 + i)^{k + 1} = (1 + i)PV(1 + i)^{k} = FV_{year = k} \times (1 + i)
        \]

        By Definition,
        \[
            FV_{year = k} \times (1 + i) = FV_{year = k + 1}
        \]

        Therefore, $FV_{year = n} = PV (1 + i)^n$

    \end{quote}

    \subsection{Future value of n year annuity with payment starting at the end of the first year}
    \subsubsection{Formula}
    \begin{quote}
        \[
            FV_{year = n} = \frac{\text{PMT}}{i}\left[(1 + i)^n - 1\right]
        \]
    \end{quote}
    \subsubsection{Derivation}
    \begin{quote}
        We will pay annuity at the end of each year from year 1 to n.

        Moreover, each annuity will compound until after n years.

        According to Formula 1.2.1, we can express the future value of the compounded kth annuity ($PV_k$) at the end of year n as
        \[
            FV_{year = n} = PV_k (1 + i)^{n - i}
        \]

        To find the total future value of n year annuity, we will combine all of the compounded kth annuity from the end of year 1 to n.
        We assume all annuity have the same PMT.
        \[
            FV_{year = n} = \sum_{k = 1}^{n} \text{PMT} (1 + i)^{k - 1} = \text{PMT} (1 + i)^{0} + \text{PMT} (1 + i)^{1} + \ldots + \text{PMT} (1 + i)^{n - 1}
        \]

        This is a divergent geometric series whose sum is
        \[
            FV_{year = n} = \frac{\text{PMT}(1 - (1 + i)^n)}{(1 - (1 + i))}
        \]

        As a result,
        \[
            FV_{year = n} = \frac{\text{PMT}}{-i}(1 - (1 + i)^n) = \boxed{\frac{\text{PMT}}{i}\left[(1 + i)^n - 1\right]}
        \]
    \end{quote}


    \subsection{Present value of n year annuity with payments start at the end of the first year}
    \subsubsection{Formula}
    \begin{quote}
        \[
            PV_{year = 0} = \frac{\text{PMT}}{i}\left[1 - \frac{1}{(1 + i)^n}\right]
        \]
    \end{quote}
    \subsubsection{Derivation}
    \begin{quote}
        The annuity allows us to withdraw the the same PMT at end of the kth year.

        We can get the present value of this withdrawal at the beginning of year 1 with
        \[
            PV_{year=0} = \frac{\text{PMT}}{(1 + i)^{k}} = \text{PMT}\left(\frac{1}{1 + i}\right)^k
        \]

        Similar to how we found Formula 1.3, we will add all of the kth present values and find the sum of the geometric series


        \[  PV_{year=0} = \sum_{k = 1}^{n} \text{PMT}\left(\frac{1}{1 + i}\right)^k = \left(\frac{1}{1 + i}\right)\left(\sum_{k = 1}^{n} \text{PMT}\left(\frac{1}{1 + i}\right)^{k - 1}\right) \]
        \[= \left(\frac{1}{1 + i}\right) \frac{\text{PMT}(1 - (\frac{1}{1 + i})^n)}{(1 - (\frac{1}{1 + i}))}\]
        \[= \left(\frac{1}{\left(1 + i - 1\right)}\right)\text{PMT}\left(1 - \left(\frac{1}{1 + i}\right)^n\right)\]
        \[= \boxed{\frac{\text{PMT}}{i}\left[1 - \frac{1}{(1 + i)^n}\right]}\]




    \end{quote}
\end{quote}

\end{document}
