\documentclass{article} % Defines the document class, article is commonly used
\usepackage[shortlabels]{enumitem}
\usepackage{amsmath}    % Allows for more advanced math formatting
\usepackage{amssymb}    % Provides additional mathematical symbols
\usepackage{amsthm}     % \qed
\usepackage{graphicx}   % image
\usepackage{float}      % image placement
\usepackage{hyperref}
\hypersetup{
    colorlinks=true,       % false: boxed links; true: colored links
    linkcolor=black,       % color of internal links
}
\usepackage[margin=1.5in]{geometry}
\usepackage{siunitx}
\newcommand{\FVA}{\mathrm{FVA}} % Future Value of Annuity
\newcommand{\PVA}{\mathrm{PVA}} % Present Value of Annuity
\newcommand{\pmt}{\text{PMT}} % Payment
\newcommand{\FV}{\mathrm{FV}} % Future Value Matrix
\newcommand{\p}{\mathcal{P}} % Principal
\newcommand{\FVSA}{\FV_{S\!A}} % Future Value Matrix for an Single-payment Annuity
\newcommand{\FVOA}{\FV_{O\!A}} % Future Value Matrix for an Ordinary Annuity

\begin{document}

\title{ARE142: Annuity As A Matrix}
\author{Tao Wang}
\date{\today}

\maketitle

\section{Introduction}
\begin{quote}

\end{quote}

\section{General Finance Definition}
\begin{quote}
    \begin{itemize}
        \item Money: a financial asset whose real worth is changing with the period according to the Time Value of Money.
        \item Payment ($\pmt$): a money. A positive payment means money added to an account, while a negative payment means it's taken out of the account.
        \item Principal ($\p$): an original sum of money invested or lent.
        \item Period: an whole number point in time that starts at 0.
        \item Future Value ($FV_{k, j}$): the worth at period $k$ of a payment made at period $j$.
        \item Present Value ($PV_j$): the worth at period 0 of a payment that is or will be made at period $j$.
              \begin{align*}
                  \\
                  PV_j & = FV_{0, j} & \text{(The Future at Right Now is Present)} \\
              \end{align*}
        \item Time Value of Money: The worth of money is constantly changing
              \begin{align*}
                  FV_{k,j} =
                  \begin{cases}
                      FV_{k-1,j} \times (1 + i_k)      & \text{if } k > j \\
                      FV_{k+1,j} \times (1 + i_k)^{-1} & \text{if } k < j
                  \end{cases}
              \end{align*}
    \end{itemize}
\end{quote}

\section{Future Value Matrix for Annuity}
\begin{quote}
    The future values ($FV_{k, j}$) of a group of payment made to an annuity (from the annuity's perspective) could be described by a Future Value Matrix (FV) that follows the Time Value of Money.
    \[
        \FV =
        \begin{bmatrix}
            FV_{0,0} & FV_{0,1} & \cdots & FV_{0,n} \\
            FV_{1,0} & FV_{1,1} & \cdots & FV_{1,n} \\
            \vdots   & \vdots   & \ddots & \vdots   \\
            FV_{n,0} & FV_{n,1} & \cdots & FV_{n,n}
        \end{bmatrix}
    \]

    where $0 \leq k \leq n$ and  $0 \leq j \leq n$
\end{quote}

\subsection{Definition: Future Value of Annuity}

\begin{quote}
    $\FVA$ is the total future value at period $k$ of all payments made to the annuity. We could think of it as summing each row of the Future Value Matrix until the diagonal element.
    \begin{align*}
        \FVA_{k} & = \sum_{j = 0}^{k} FV_{k, j} & (\text{Future Value of Annuity}) \\
    \end{align*}
\end{quote}

\subsection{Definition: Present Value of Annuity}
\begin{quote}
    $\PVA$ is the total present value at period 0 of all payments that is or will be made to the annuity.
    \begin{align*}
        \PVA & = \sum_{j = 0}^{n} PV_{j} = \sum_{j = 0}^{n} FV_{0, j} & (\text{Present Value of Annuity}) \\
    \end{align*}
\end{quote}

\section{Types of Annuity}
\begin{quote}
    There are different types of annuity that each fill the Future Value Matrix in a different way.

    Using the Definition of each Annuity, we can derive the Growth Function and the Valuation. The Growth Function encapsulates the recursive relationship between sums of each row of the Future Value Matrix. The Valuation is a closed form formula for the sum of row $k$.
\end{quote}

\subsection{Single Payment Annuity}
\begin{quote}
    A Single Payment Annuity is denoted by $\FVSA$.

    It is A single payment, $\pmt$, made to the annuity at period 0.
\end{quote}
\subsubsection{Definition}
\begin{align*}
     & FV_{k,j}        & =
    \begin{cases}
        \p                               & \text{if } k = 0 \text{ and } j = 0    \\
        0                                & \text{if } k \neq 0 \text{ and } k = j \\
        FV_{k-1,j} \times (1 + i_k)      & \text{if } k > j                       \\
        FV_{k+1,j} \times (1 + i_k)^{-1} & \text{if } k < j                       \\
    \end{cases} \\\\
     & 0 \leq k \leq n & \text{(Definition of Future Value Matrix)}           \\
     & 0 \leq j \leq n & \text{(Definition of Future Value Matrix)}           \\
     & i_k = i_0       & \text{(Constant Interest)}                           \\
\end{align*}
\[
    \FVSA =
    \begin{bmatrix}
        \p            & 0      & \cdots & 0      \\
        \p(1 + i_0)^1 & 0      & \cdots & 0      \\
        \vdots        & \vdots & \ddots & \vdots \\
        \p(1 + i_0)^n & 0      & \cdots & 0      \\
    \end{bmatrix}
\]
\subsubsection{Growth Function}
\[\FVA_{k+1} = \FVA_{k} \times (1 + i_0)\]

\subsubsection{Valuation}
\begin{align*}
     & \PVA = \FVA_0 = \p        \\
     & \FVA_k = \p (1 + i_0) ^ k \\
\end{align*}
\subsubsection{Derivation}
\begin{quote}
    Derive $\FVA_{k}$
\end{quote}

\subsection{Ordinary Annuity}
\begin{quote}
    An Ordinary Annuity is represented by $\FVOA$.

    It is a regular stream of equal payments to the annuity starting at period $1$ to period $n$ (inclusive) with a principal, $\p$.
\end{quote}
\subsubsection{Definition}
\begin{align*}
     & FV_{k,j}        & =
    \begin{cases}
        \p                               & \text{if } k = 0 \text{ and } j = 0    \\
        \pmt                             & \text{if } k \neq 0 \text{ and } k = j \\
        FV_{k-1,j} \times (1 + i_k)      & \text{if } k > j                       \\
        FV_{k+1,j} \times (1 + i_k)^{-1} & \text{if } k < j                       \\
    \end{cases} \\\\
     & 0 \leq k \leq n & \text{(Definition of Future Value Matrix)}           \\
     & 0 \leq j \leq n & \text{(Definition of Future Value Matrix)}           \\
     & i_k = i_0       & \text{(Constant Interest)}                           \\
\end{align*}
\[
    \FVOA =
    \begin{bmatrix}
        \p            & \pmt (1 + i_0)^{-1}  & \cdots & \pmt (1 + i_0)^{-n}       \\
        \p(1 + i_0)^1 & \pmt                 & \cdots & \pmt (1 + i_0)^{-(n - 1)} \\
        \vdots        & \vdots               & \ddots & \vdots                    \\
        \p(1 + i_0)^n & \pmt (1 + i_0)^{n-1} & \cdots & \pmt                      \\
    \end{bmatrix}
\]
\subsubsection{Growth Function}
\[\FVA_{k+1} = \FVA_{k} \times (1 + i_0) + \pmt\]

\subsubsection{Valuation}
\begin{align*}
    \PVA   & = \p + \sum_{j = 1}^{n} \pmt \left( \frac{1}{1+i_0} \right)^j            \\
           & = \p + \frac{\pmt}{i_0}\left[1 - \frac{1}{(1 + i_0)^n}\right]            \\
    \FVA_k & = \p(1+i_0)^k + \sum_{k = 1}^{n} \pmt (1 + i)^{k - 1}                    \\
           & =\p(1+i_0)^k + \displaystyle\frac{\pmt}{i_0}\left[(1 + i_0)^k - 1\right] \\
\end{align*}
\subsubsection{Derivation}
\begin{quote}
    Derive $\FVA_{k}$
\end{quote}



\end{document}
